\section{Stability Analysis}

We investigate the stabiliy of the Crank-Nicolson method using the equation
from \eqref{eq:prey}. A similar argument may be used to show the stability of
the Crank-Nicolson method using the equation from \eqref{eq:pred}.

\begin{align} \label{eq:stab_initial}
\frac{U_n^{m+1} - U_n^m}{k} &=
    \frac{D_u}{2} \left(
        \frac{U_{n-1}^{m+1} - 2 U_{n}^{m+1} + U_{n+1}^{m+1}}{h^2} +
        \frac{U_{n-1}^{m} - 2 U_{n}^{m} + U_{n+1}^{m}}{h^2}
    \right)
\end{align}

We then substitute $U_n^m = \xi^m e^{iknh}$ and rearrange the terms in
\eqref{eq:stab_initial} such that:

% Can't split large brackets over multiple lines so we create dummy brackets.
\begin{align} \label{eq:stab_subbed}
\begin{split}
\xi^{m+1} e^{iknh} - \xi^{m} e^{iknh} =
    \frac{D_u k}{2h^2} \left(
        \xi^{m+1} e^{ik(n+1)h} + \xi^{m} e^{ik(n+1)h} \right. \\
        - 2 \xi^{m+1} e^{iknh} - 2 \xi^{m} e^{iknh} \\
        \left. + \xi^{m+1} e^{ik(n-1)h} + \xi^{m} e^{ik(n-1)h} \right)
\end{split}
\end{align}

We may then divide throughout by $\xi^{m} e^{iknh}$ which gives the following:

\begin{align} \label{eq:stab_divided}
\xi - 1 &= \frac{D_u k}{2h^2} \left(
    \xi e^{ikh} + e^{ikh} - 2 \xi - 2 + \xi e^{-ikh} + e^{-ikh}
    \right)
\end{align}

Rearranging \eqref{eq:stab_divided} and replacing the complex exponential gives:

\begin{align} \label{eq:stab_trig}
\xi &= 1  - \frac{D_u \tau}{2h^2} (\xi + 1) [1 - \cos(kh)]
\end{align}

Taking the absolute value of $\xi$ gives:

\begin{align} \label{eq:stab_abs}
| \xi | &= \left|
    \frac{1 - D_u \tau / h^2 [1 - \cos(kh)]}
         {1 + D_u \tau / h^2 [1 - \cos(kh)]}
    \right| \leq 1
\end{align}

Since \eqref{eq:stab_abs} is true for any value of $k$ the Crank-Nicolson model
is unconditionally stable.
