\section{Derivation}

From the problem statement we have the following system of PDEs.

\begin{subequations} \label{eq:system}
\begin{align}
    u_t &= D_u u_{xx} + f(u,v) \label{eq:prey} \\
    v_t &= D_v v_{xx} + g(u,v) \label{eq:pred}
\end{align}
\end{subequations}

where

\begin{subequations}
\begin{align*}
    f(u,v) &= u(1-u) - \frac{auv}{1+\lambda u} \\
    g(u,v) &= - \frac{v}{ab} + \frac{auv}{b(1+\lambda u)}
\end{align*}
\end{subequations}

We use the following finite difference approximations, derived from the
Taylor series expansion, in place of the $u_t$, $u_{xx}$, $v_t$, and
$v_{xx}$ in equations \eqref{eq:system}.

\begin{subequations} \label{eq:taylor}
\begin{align}
    u_t &= \frac{u(x,t+k)-u(x,t)}{k} - \frac{k}{2} u_{tt}(x,\tau) \label{eq:u_t} \\
    v_t &= \frac{v(x,t+k)-v(x,t)}{k} - \frac{k}{2} v_{tt}(x,\sigma) \label{eq:v_t}
    \\
    u_{xx} &= \frac{u(x-h,t)-2u(x,t)+u(x+h,t)}{h^2} - \frac{h^2}{12} u_{xxxx}(\xi,t) \label{eq:u_xx} \\
    v_{xx} &= \frac{v(x-h,t)-2v(x,t)+v(x+h,t)}{h^2} - \frac{h^2}{12} v_{xxxx}(\epsilon,t) \label{eq:v_xx}
\end{align}
\end{subequations}

By discarding the truncation terms from equations \eqref{eq:taylor} and letting
$U_{i,j}$ and $V_{i,j}$ approximate the true solutions $u(x_i,t_j)$ and
$v(x_i,t_j)$ respectively we can replace equations \eqref{eq:system} with the
following system of equations. We write $f(u_i^j,v_i^j)) = f_i^j$ and
$g(u_i^j,v_i^j)) = g_i^j$.

\begin{subequations} \label{eq:approx_sys}
\begin{align}
    \frac{U_{i}^{j+1} - U_{i}^{j}}{k} &=
        \frac{D_u}{2} \left\{
            \left[ \frac{U_{i-1}^{j+1}-2U_{i}^{j+1}+U_{i+1}^{j+1}}{h^2} \right] +
            \left[ \frac{U_{i-1}^{j}-2U_{i}^{j}+U_{i+1}^{j}}{h^2} \right]
        \right\} + f_{i}^{j} \label{eq:approx_prey}
        \\
    \frac{V_{i}^{j+1} - V_{i}^{j}}{k} &=
        \frac{D_v}{2} \left\{
            \left[ \frac{V_{i-1}^{j+1}-2V_{i}^{j+1}+V_{i+1}^{j+1}}{h^2} \right] +
            \left[ \frac{V_{i-1}^{j}-2V_{i}^{j}+V_{i+1}^{j}}{h^2} \right]
        \right\} + g_{i}^{j} \label{eq:approx_pred}
\end{align}
\end{subequations}

Multiplying equations \eqref{eq:approx_sys} throughout by $2k$ and rearranging
the terms yields the following with $r = D_u k / h^2$ and $s = D_v k / h^2$.

\begin{subequations} \label{eq:rearranged_approx_sys}
\begin{align}
    (2+2r) U_{i}^{j+1} - r U_{i-1}^{j+1} - r U_{i+1}^{j+1} &=
        (2-2r) U_{i}^{j} + r U_{i-1}^{j} + r U_{i+1}^{j} + 2k f_{i}^{j}
        \\
    (2+2s) V_{i}^{j+1} - s V_{i-1}^{j+1} - s V_{i+1}^{j+1} &=
        (2-2s) V_{i}^{j} + s V_{i-1}^{j} + s V_{i+1}^{j} + 2k g_{i}^{j}
\end{align}
\end{subequations}


We now investigate the boundary conditions. When $i=0$ we have may use the
central difference formula for the first derivatives as follows:

\begin{subequations}
\begin{align}
    \frac{\partial U_{0}^{j+1}}{\partial x} &\approx \frac{{U_1}^{j+1} - U_{-1}^{j+1}}{2h} \\
    \frac{\partial V_{0}^{j+1}}{\partial x} &\approx \frac{{V_1}^{j+1} - V_{-1}^{j+1}}{2h}
\end{align}
\end{subequations}

Since the derivatives at the boundaries are given as 0 this implies that:

\begin{subequations}
\begin{align}
    U_{1}^{j+1} &= U_{-1}^{j+1} \\
    V_{1}^{j+1} &= V_{-1}^{j+1}
\end{align}
\end{subequations}

The same argument follows for the right endpoint where $i=N$ and we have:

\begin{subequations}
\begin{align}
    U_{N-1}^{j+1} &= U_{N-1}^{j+1} \\
    V_{N+1}^{j+1} &= V_{N-1}^{j+1}
\end{align}
\end{subequations}

We may then substitute these into their respective equations so that:

\begin{subequations} \label{eq:rearranged_approx_sys_endpoints}
\begin{align}
    (2+2r) U_{0}^{j+1} - 2r U_{1}^{j+1} &=
        (2-2r) U_{0}^{j} + 2r U_{1}^{j} + 2k f_{i}^{j}
        \\
    (2+2s) V_{0}^{j+1} - 2s V_{1}^{j+1} &=
        (2-2s) V_{0}^{j} + 2s V_{1}^{j} + 2k g_{i}^{j}
        \\
    (2+2r) U_{N}^{j+1} - 2r U_{N-1}^{j+1} &=
        (2-2r) U_{N}^{j} + 2r U_{N-1}^{j} + 2k f_{i}^{j}
        \\
    (2+2s) V_{N}^{j+1} - 2s V_{N-1}^{j+1} &=
        (2-2s) V_{N}^{j} + 2s V_{N-1}^{j} + 2k g_{i}^{j}
\end{align}
\end{subequations}

These systems may be rewritten into the following matrix equations:

\begin{align}
\begin{bmatrix}
    2+2r   & -2r    & \dots  & 0 \\
    -r     & 2+2r   & \dots  & 0 \\
    0      & -r     & \ddots & \vdots \\
    \vdots & \vdots & \ddots & -r   \\
    0      & 0      & -2r    & 2+2r
\end{bmatrix}
\begin{bmatrix}
    f_1^{j+1} \\
    f_2^{j+1} \\
    f_3^{j+1} \\
    \vdots    \\
    f_n^{j+1}
\end{bmatrix}
&=
\begin{bmatrix}
    2-2r   & 2r     & \dots  & 0 \\
    r      & 2-2r   & \dots  & 0 \\
    0      & r      & \ddots & \vdots \\
    \vdots & \vdots & \ddots & r   \\
    0      & 0      & 2r     & 2-2r
\end{bmatrix}
\begin{bmatrix}
    f_1^{j} \\
    f_2^{j} \\
    f_3^{j} \\
    \vdots    \\
    f_n^{j}
\end{bmatrix} \\[10pt]
\begin{bmatrix}
    2+2s   & -2s    & \dots  & 0 \\
    -s     & 2+2s   & \dots  & 0 \\
    0      & -s     & \ddots & \vdots \\
    \vdots & \vdots & \ddots & -s   \\
    0      & 0      & -2s    & 2+2s
\end{bmatrix}
\begin{bmatrix}
    g_1^{j+1} \\
    g_2^{j+1} \\
    g_3^{j+1} \\
    \vdots    \\
    g_n^{j+1}
\end{bmatrix}
&=
\begin{bmatrix}
    2-2s   & 2s     & \dots  & 0 \\
    s      & 2-2s   & \dots  & 0 \\
    0      & s      & \ddots & \vdots \\
    \vdots & \vdots & \ddots & s   \\
    0      & 0      & 2s     & 2-2s
\end{bmatrix}
\begin{bmatrix}
    g_1^{j} \\
    g_2^{j} \\
    g_3^{j} \\
    \vdots    \\
    g_n^{j}
\end{bmatrix}
\end{align}
